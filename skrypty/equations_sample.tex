Treść akapitu i równanie                 % Treść akapitu
\begin{gather}                           % Rozpocznij równanie
f_{2}(x) = 123 x + 321 \label{eq:2}, \\  % Etykieta i nowa linia
f_{3}(x) = 8 x + 7     \label{eq:3},     % Etykieta równania
\end{gather}                             % Zakończ równanie
kontynuacja treści akapitu               % Treść akapitu
\begin{align}                            % Rozpocznij równanie
f_{4}(x) &= 896 x - 777 \label{eq:4}, \\ % Etykieta i nowa linia
f_{5}(x) &= 2 x + 3     \label{eq:5}.    % Etykieta równania
\end{align}                              % Zakończ równanie
Kolejna treść akapitu                    % Treść akapitu
\begin{equation}                         % Rozpocznij równanie
f_{6}(x) =                               % Poczatek równania
\begin{cases}                            % Początek bloku 'cases'
0, & \text{jeżeli $x \ge \pi$}  \\       % Przypadek pierwszy
1, & \text{w przeciwnym razie}           % Przypadek drugi
\end{cases}                              % Koniec bloku 'cases'
\label{eq:6}.                            % Etykieta równania
\end{equation}                           % Zakończ równanie
                                         % Podwójna nowa linia
Treść nowego akapitu...                  % Nowy akapit
