\begin{table}[htb]
\pgfplotstabletypeset[
	header = has colnames,                 % Używaj nagłówka
	col sep = semicolon,                   % Ustaw separator
	sort, sort key = unc,                  % Załącz sortowanie
	columns = {nq, xmax, xmin, nvar, unc}, % Lista używanych kolumn
	columns/nq/.style = {    % Ustawienia kolumny 'nq'
		column name = \textbf{$n_q$},
		dec sep align          % Wyrównaj wartości do separatora
	},
	columns/xmax/.style = {  % Ustawienia kolumny 'xmax'
		column name = \textbf{$x_{\max}$, \unit{V}},
		fixed, fixed zerofill, % Format stałoprzecinkowy, dopełniony
		precision = 2,         % zerami, dwa miejsca po przecinku
	},
	columns/xmin/.style = {  % Ustawienia kolumny 'xmin'
		column name = \textbf{$x_{\min}$, \unit{V}},
		fixed, fixed zerofill, % Format stałoprzecinkowy, dopełniony
		precision = 2,         % zerami, dwa miejsca po przecinku
	},
	columns/nvar/.style = {  % Ustawienia kolumny 'nvar'
		column name = \textbf{$\sigma_n^2$, \unit{mV^2}},
		fixed, fixed zerofill, % Format stałoprzecinkowy, dopełniony
		precision = 1,         % zerami, jedno miejsce po przecinku
		multiply by = 1000,    % Pomnóż wartości razy tysiąc
	},
	columns/unc/.style = {   % Ustawienia kolumny 'unc'
		column name = \textbf{Niepewność, \unit{mV}},
		fixed,                 % Format stałoprzecinkowy, bez zer
		precision = 0,         % oraz bez części dziesiętnej
		multiply by = 1000,    % Pomnóż wartości razy tysiąc
	}
]{nazwa_pliku_z_danymi}    % Wczytaj dane z pliku
\caption{Podpis tabeli}    % Dodaj podpis
\label{nazwa_tabeli}       % Utwórz etykietę
\end{table}
