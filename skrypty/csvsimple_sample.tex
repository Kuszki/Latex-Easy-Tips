\begin{table}[htb]
\csvreader[
	head to column names,  % Załącz przetwarzanie nazw kolumn
	separator = semicolon, % Ustaw separator kolumn na średnik
	tabular = |c|c|c|c|c|, % Ustal wyrównanie i linie dla kolumn
	table head =           % Utwórz nagłówek dla tabeli:
		\hline                                % Linia pozioma
		\textbf{$n_q$}                     &  % Kolumna 'nq'
		\textbf{$x_{\max}$, \unit{V}}      &  % Kolumna 'xmax'
		\textbf{$x_{\min}$, \unit{V}}      &  % Kolumna 'xmin'
		\textbf{$\sigma_n^2$, \unit{mV^2}} &  % Kolumna 'nvar'
		\textbf{Niepewność, \unit{mV}}     \\ % Kolumna 'unc'
		\hline,                               % Linia pozioma
	late after line = \\ \hline, % Dodawaj linię po każdym wierszu
]{wyniki_eksperymentu.csv}{}{  % Wczytuj dane z pliku tekstowego
	\tablenum[table-format =  4.0, round-precision = 0]{\nq}   & 
	\tablenum[table-format = +2.2, round-precision = 2]{\xmax} & 
	\tablenum[table-format = +2.2, round-precision = 2]{\xmin} & 
	\tablenum[table-format =  3.2, round-precision = 2]
		{\fpeval{\nvar * 1000}} & 
	\tablenum[round-mode = figures, round-precision = 2]
		{\fpeval{\unc * 1000}}
}
\caption{Podpis tabeli}       % Dodaj podpis do tabeli
\label{nazwa_tabeli}          % Dodaj etykietę tabeli
\end{table}
