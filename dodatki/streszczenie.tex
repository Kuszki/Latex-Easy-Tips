\chapter*{Streszczenie}\addcontentsline{toc}{chapter}{Streszczenie}

Pracę poświęcono opisowi wybranych narzędzi informatycznych, przydatnych podczas opisu realizowanych badań naukowych, wykonywaniu sprawozdań oraz przygotowywania dokumentów. Praca nie jest typową instrukcją stosowania omówionych w niej narzędzi, a jej celem jest raczej uświadomienie czytelnika o ich istnieniu i zachęta do ich wypróbowania. W pracy omówiono podstawy korzystania z systemu składu \LaTeX{}, zaproponowano w jaki sposób archiwizować wyniki badań i wykorzystywać se do sporządzania treści dokumentu, omówiono w jaki sposób tworzyć spójne ze stylem dokumentu rysunki oraz jak wstawiać do niego fragmenty skryptów. Wszystkie omówione w pracy narzędzia stanowią wolne oprogramowanie. Głównym celem pracy jest popularyzacja stosowania narzędzi, które w opinii autora bardzo ułatwiają codzienną pracę i zwiększają jakość składu tworzonych publikacji.
