\chapter*{Streszczenie}\addcontentsline{toc}{chapter}{Streszczenie}

Pracę poświęcono opisowi wybranych narzędzi informatycznych, przydatnych podczas dokumentacji i realizacji badań naukowych, wykonywaniu sprawozdań oraz przygotowywania manuskryptów. Praca nie jest typową instrukcją stosowania omówionych w niej narzędzi, a jej głównym celem jest uświadomienie czytelnika o ich istnieniu i zachęta do ich stosowania. W pracy omówiono podstawy korzystania z systemu składu \LaTeX{}, zaproponowano w jaki sposób archiwizować wyniki badań i wykorzystywać je do sporządzania treści dokumentu, omówiono w jaki sposób tworzyć spójne ze stylem dokumentu rysunki oraz jak wstawiać do niego fragmenty skryptów. Głównym celem pracy jest popularyzacja stosowania narzędzi, które w opinii autora bardzo ułatwiają codzienną pracę i zwiększają jakość składu tworzonych publikacji. Wszystkie omówione w pracy narzędzia stanowią wolne oprogramowanie, dostępne dla większości systemów operacyjnych i platform sprzętowych.

Niniejsza praca dostępna jest na warunkach licencji \href{https://creativecommons.org/licenses/by-sa/4.0}{CC BY-SA 4.0}. Dozwolone jest rozpowszechnianie, modyfikowanie, dostosowywanie niniejszej pracy i tworzenie materiałów na dowolnym nośniku lub w dowolnym formacie, również w zastosowaniach komercyjnych, pod warunkiem wskazania twórcy oryginalnej pracy oraz zachowania identycznych warunków licencji na dzieło pochodne. Praca oraz jej kod źródłowy dostępne są publicznie w serwisie \texttt{GitHub}~\cite{auth_this}. Praca nie została zrecenzowana i stanowi zbiór osobistych przemyśleń autora.
