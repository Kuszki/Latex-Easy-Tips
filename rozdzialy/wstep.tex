\chapter{Wstęp do pracy}

Praca związana z prowadzeniem badań naukowych składa się z kilku istotnych etapów. Poza najważniejszymi, do których należą procesy myślowe, praca ta wiąże się również z wykonywaniem różnych eksperymentów, pozyskiwaniem danych pomiarowych, analizą tych danych oraz ostatecznie opisem uzyskanych wyników. Końcowy odbiorca owoców takiej pracy ma najczęściej styczność jedynie z dokumentem, w którym autor zawarł opis i wyniki przeprowadzonych badań. Jest to najczęściej stosowny raport lub artykuł w czasopiśmie. Oznacza to, że manuskrypt ten musi zostać przygotowany równie starannie, co pozostałe etapy pracy badawczej.

Można w tym miejscu zauważyć, że o ile poprawny i odpowiedni opis merytoryczny zawarty w dokumencie jest bardzo istotny i nie ma większych możliwości automatyzacji takiego opisu, o tyle w przypadku procesu wizualizacji danych i ich obróbki posiłkować się można różnorakimi narzędziami. Sam dokument musi być również odpowiednio sformatowany, gdzie w przypadku artykułu naukowego to wydawnictwo narzuca zasady jego formatowania. Warto w tym miejscu zauważyć, że dokumenty związane z dziedziną nauk technicznych zawierają zwykle bardzo dużo wzorów, symboli osadzonych w tekście, tabel z danymi i rysunków. Oznacza to w praktyce, że autor poświecą ogromną ilość czasu na formatowanie takiego dokumentu, popełnia przy tym wiele typowo ludzkich błędów oraz często doświadcza frustracji związanej z niedoskonałościami stosowanego edytora tekstu.

Problemy z formatowaniem dokumentu nawarstwiają się dodatkowo w przypadku, gdy jego struktura ulega zmianie. Oczywiście pewien poziom automatyzacji, taki jak automatyczne numerowanie równań, rysunków, tabel, czy pozycji bibliografii, osiągnąć można bez problemu stosując klasyczny edytor tekstu. W praktyce jednak w przypadku kilkustronicowych artykułów bardzo niewielu autorów decyduje się na takie rozwiązanie, wprowadzając numerację ręcznie. Kolejne wyzwanie stanowi klasyczna sytuacja, w której ten sam dokument wygląda inaczej w innej wersji edytora, albo \enquote{rozpada się} po wykonaniu pewnej akcji, a następnie nie wraca właściwej formy po jej wycofaniu. Istotnym problemem jest również skopiowanie fragmentu tekstu lub równań do dokumentu o innym stylu lub ekspresowa zmiana formatu w przypadku wyboru innego czasopisma, do którego wysyłany jest manuskrypt.

Podobne problemy spotkać można również podczas przygotowywania wykresów oraz sporządzania tabel. Bardzo często w roboczej wersji manuskryptu dostrzega się pewne błędy. Niejednokrotnie po ich korekcji konieczne jest wygenerowanie wykresu lub wstawienie do tabeli danych od nowa. Wiąże się to z kolejną mozolną pracą, tym bardziej że surowe dane, które są obliczane przez stosowany w tym celu program, wymagają często odpowiedniego formatowania. Bardzo często dochodzi również do sytuacji, gdzie wykresy lub obrazki osadzane są w formacie rastrowym, krój i rozmiar czcionki nie zgadzają się ze stylem dokumentu, a ich jakość jest wątpliwa.

Powyższe wyzwania są elementem codziennej pracy, stąd w opinii autora istotne jest opanowanie odpowiednich narzędzi, które ułatwią i zautomatyzują ich pokonywanie. Inicjatywa ta zaowocuje ogromną oszczędnością czasu, przez co naukowiec będzie mógł skupić się na istotnych elementach swojej pracy. Poza zwiększeniem jakości sporządzanych dokumentów spadnie również poziom frustracji, spowodowany \enquote{walką} z mozolnymi czynnościami i niedoskonałością klasycznego edytora tekstu. Należy zauważyć, że człowiek bardzo często przyzwyczaja się do pewnych rozwiązań, uznając związane z nimi niedogodności za nieuniknione. Brak świadomości istnienia dużo lepszych narzędzi, wyspecjalizowanych w danych czynnościach, ogranicza możliwości jego rozwoju. Z drugiej strony przyzwyczajenie to może skutkować również blokadą i niechęcią do zapoznania się z nowymi rozwiązaniami. Warto jednak czasami poświęcić chwilę i zainwestować cenny czas w naukę nowych rozwiązań. Być może bowiem inwestycja ta przyczyni się do istotnych oszczędności czasu w przyszłości.

Pracę podzielono na dwa główne rozdziały. Pierwszy z nich poświęcony jest wprowadzeniu do systemu składu \LaTeX{}. W rozdziale tym skupiono się na scenariuszu, gdzie zadaniem użytkownika jest sporządzenie pracy wykorzystując gotowy szablon, dostarczany przez wydawnictwo czasopisma. Omówione zostały podstawy pracy z dokumentem, proces jego kompilacji, sporządzanie wzorów, osadzanie rysunków, fragmentów kodu źródłowego oraz tabel, formatowanie liczb i wielkości fizycznych oraz zarządzanie bibliografią. Rozdział drugi poświęcony jest nieco bardziej skomplikowanym zagadnieniom, które związane są z automatyzacją pracy. Omówiono w nim, między innymi, w jaki sposób wykonywać obliczenia i archiwizować ich wyniki, a także jak automatycznie wstawiać je do dokumentu. W rozdziale tym przedstawiono kilka technik umożliwiających automatyczne tworzenie tabel i wykresów w ten sposób, aby ich zawartość nie musiała być ręcznie wklejana i formatowana przez użytkownika.
