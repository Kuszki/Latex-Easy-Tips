\chapter{Wstęp do pracy}

Praca związana z prowadzeniem badań naukowych składa się z kilku istotnych etapów. Poza najważniejszymi, do których należą procesy myślowe, praca ta wiąże się również z wykonywaniem różnych eksperymentów, pozyskiwaniem danych pomiarowych, analizą tych danych oraz ostatecznie opisem uzyskanych wyników. Ostateczny odbiorca owoców takiej pracy ma najczęściej styczność jedynie z dokumentem, w którym autor badań zawarł ich opis. Jest to zatem stosowny raport lub artykuł w czasopiśmie. Oznacza to, że ostateczny owoc pracy musi zostać przygotowany równie starannie, co pozostałe etapy pracy badawczej.

Można w tym miejscu zauważyć, że o ile poprawny i odpowiedni opis merytoryczny zawarty w ostatecznym dokumencie jest bardzo istotny i nie ma większych możliwości automatyzacji takiego opisu, o tyle w przypadku procesu wizualizacji danych i ich obróbki posiłkować się można różnorakimi narzędziami. Sam dokument musi być również odpowiednio sformatowany, gdzie w przypadku artykułu naukowego to wydawnictwo narzuca zasady jego formatowania. Warto w tym miejscu zauważyć, że dokumenty związane z dziedziną nauk technicznych zawierają zwykle bardzo dużo wzorów, symboli osadzonych w tekście, tabel z danymi i rysunków. Oznacza to w praktyce, że autor poświecą bardzo dużo czasu na formatowanie takiego dokumentu, popełnia przy tym wiele ludzkich błędów oraz doświadcza często frustracji związanej ze stosowaniem edytora tekstu.

Problemy z formatowaniem dokumentu nawarstwiają się dodatkowo w przypadku, gdy jego struktura ulega zmianie. Oczywiście pewien poziom automatyzacji, taki jak automatyczne numerowanie równań, rysunków, tabel, wpisów bibliograficznych, osiągnąć można stosując klasyczny edytor tekstu. W praktyce jednak w przypadku kilkustronicowych artykułów bardzo niewielu autorów decyduje się na takie rozwiązanie. Kolejne wyzwanie stanowi klasyczna sytuacja, w której ten sam dokument wygląda inaczej w innej wersji edytora, albo \enquote{rozpada się} po wykonaniu pewnej akcji, przy czym \enquote{nie wraca do ładu} po jej wycofaniu. Problemem nie do przejścia jest również skopiowanie fragmentu tekstu lub równań do innego dokumentu o innym stylu lub ekspresowa zmiana stylu w przypadku zmiany czasopisma, do którego wysyłany jest manuskrypt.

Podobne problemy spotkać można również podczas przygotowywania wykresów oraz sporządzania tabel. Bardzo często w roboczej wersji manuskryptu dostrzega się pewne błędy. Niejednokrotnie po ich korekcji konieczne jest wygenerowanie wykresu lub wstawienie do tabeli danych od nowa. Wiąże się to z kolejną mozolną pracą, tym bardziej że surowe dane, które są obliczane przez stosowany w tym celu program, wymagają często odpowiedniego formatowania. Bardzo często dochodzi również do sytuacji, gdzie wykresy lub obrazki osadzane są w formacie rastrowym, krój i rozmiar czcionki nie zgadzają się ze stylem dokumentu, a ich jakość jest wątpliwa.

Powyższe wyzwania są elementem codziennej pracy, stąd w opinii autora niniejszej pracy istotne jest opanowanie odpowiednich narzędzi, które ułatwią i zautomatyzują ich pokonywanie. Działanie to zaowocuje ogromną oszczędnością czasu, przez co naukowiec będzie mógł skupić się na istotnych elementach swojej pracy. Poza zwiększeniem jakości sporządzanych dokumentów spadnie również poziom zmęczenia i frustracji, spowodowany \enquote{walką} z mozolnymi czynnościami i niedoskonałością klasycznego edytora tekstu.

Niniejsza praca dostępna jest na warunkach licencji \href{https://creativecommons.org/licenses/by-sa/4.0}{CC BY-SA 4.0}. Dozwolone jest rozpowszechnianie, modyfikowanie, dostosowywanie niniejszej pracy i tworzenie materiałów na dowolnym nośniku lub w dowolnym formacie, również w zastosowaniach komercyjnych, pod warunkiem wskazania twórcy oryginalnej pracy oraz zachowania identycznych warunków licencji na dzieło pochodne. Praca oraz jej kod źródłowy dostępne są publicznie w serwisie \texttt{GitHub}~\cite{auth_this}. Praca nie została zrecenzowana i stanowi zbiór osobistych przemyśleń autora.
