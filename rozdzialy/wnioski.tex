\chapter{Podsumowanie pracy}

Niniejsza praca stanowi tylko i wyłącznie pewien wykaz narzędzi, które stosować można w celu usprawnienia codziennej pracy. Ideą pracy nie było bowiem stworzenie instrukcji obsługi tych narzędzi, lecz uświadomienie odbiorcy o ich istnieniu. Bardzo często aż do chwili, kiedy człowiek nie uświadomi sobie o istnieniu jakiegoś narzędzia, jest w pełni zadowolony ze sposobu w jaki radzi sobie z codziennymi wyzwaniami. Ten brak świadomości, przy jednoczesnym braku motywacji do szukania nowych rozwiązań, bardzo skutecznie hamuje jego rozwój i tworzy niepotrzebne ograniczenia.

Zrozumiałe jest, że przyzwyczajenia i utarte metody pracy są trudne do zmiany, tym bardziej jeśli próg wejścia do stosowania nowych narzędzi jest wysoki. Należy jednak w pierwszej kolejności zapoznać się z możliwymi korzyściami i dopiero wtedy ocenić, czy opłaca się poświęcić czas na naukę nowych rozwiązań. Podczas nauki najlepiej posiłkować się gotowymi przykładami oraz najprostszymi rozwiązaniami, aby nie zniechęcić się zbyt szybko porażką. Początkowo ta jest nieuniknionym doświadczeniem nawet podczas prostych czynności.

Autor pracy ma nadzieję, że jej czytelnik przynajmniej wypróbuje część z opisanych w niej rozwiązań. Naturalnie, najczęściej \enquote{right now} jest ważniejsze niż \enquote{right way}. Można jednak w wielu przypadkach zauważyć, że zrobienie czegoś \enquote{raz, a porządnie} jest szybsze od ciągłego robienia tego \enquote{byle jak} i wiecznego poprawiania rezultatu.
